\documentclass[12pt,a4paper]{article}
\usepackage[utf8]{inputenc}
\usepackage[vietnamese]{babel}
\usepackage{geometry}
\usepackage{graphicx}
\usepackage{tikz}
\usepackage{longtable}
\usepackage{array}
\usepackage{booktabs}
\usepackage{listings}
\usepackage{xcolor}
\usepackage{hyperref}
\usepackage{fancyhdr}
\usepackage{titlesec}
\usepackage{enumitem}
\usepackage{float}

\usetikzlibrary{shapes.geometric, arrows, positioning, fit, backgrounds, calc}

\geometry{margin=2.5cm}

% Colors
\definecolor{clientcolor}{RGB}{52, 152, 219}
\definecolor{servercolor}{RGB}{46, 204, 113}
\definecolor{msgcolor}{RGB}{241, 196, 15}
\definecolor{notifycolor}{RGB}{155, 89, 182}
\definecolor{codegreen}{RGB}{0,128,0}
\definecolor{codegray}{RGB}{128,128,128}
\definecolor{codeblue}{RGB}{0,0,255}
\definecolor{backcolour}{RGB}{245,245,245}

% TikZ Styles
\tikzstyle{startstop} = [rectangle, rounded corners, minimum width=3cm, minimum height=1cm, text centered, draw=black, fill=red!30]
\tikzstyle{process} = [rectangle, minimum width=3cm, minimum height=1cm, text centered, draw=black, fill=orange!30]
\tikzstyle{decision} = [diamond, minimum width=3cm, minimum height=1cm, text centered, draw=black, fill=green!30, aspect=2]
\tikzstyle{io} = [trapezium, trapezium left angle=70, trapezium right angle=110, minimum width=2cm, minimum height=1cm, text centered, draw=black, fill=blue!20]
\tikzstyle{arrow} = [thick,->,>=stealth]
\tikzstyle{client} = [rectangle, rounded corners, minimum width=2cm, minimum height=0.8cm, draw=clientcolor, fill=clientcolor!20, font=\small]
\tikzstyle{server} = [rectangle, rounded corners, minimum width=2cm, minimum height=0.8cm, draw=servercolor, fill=servercolor!20, font=\small]
\tikzstyle{msg} = [rectangle, minimum width=4cm, minimum height=0.6cm, draw=msgcolor, fill=msgcolor!20, font=\scriptsize]
\tikzstyle{notify} = [rectangle, minimum width=4cm, minimum height=0.6cm, draw=notifycolor, fill=notifycolor!20, font=\scriptsize]

% Listings style for JSON
\lstdefinestyle{jsonstyle}{
    backgroundcolor=\color{backcolour},
    basicstyle=\ttfamily\footnotesize,
    breakatwhitespace=false,
    breaklines=true,
    captionpos=b,
    keepspaces=true,
    numbers=left,
    numbersep=5pt,
    numberstyle=\tiny\color{codegray},
    showspaces=false,
    showstringspaces=false,
    showtabs=false,
    tabsize=2,
    frame=single,
    framerule=0.5pt,
    rulecolor=\color{gray}
}

\lstset{style=jsonstyle}

% Header/Footer
\pagestyle{fancy}
\fancyhf{}
\fancyhead[L]{\leftmark}
\fancyhead[R]{English Learning App}
\fancyfoot[C]{\thepage}
\renewcommand{\headrulewidth}{0.4pt}

% Title formatting
\titleformat{\section}{\Large\bfseries}{\thesection}{1em}{}
\titleformat{\subsection}{\large\bfseries}{\thesubsection}{1em}{}

\begin{document}

% Title Page
\begin{titlepage}
    \centering
    \vspace*{2cm}

    {\Huge\bfseries Quy Trình Chức Năng\\[0.3cm] và Bản Tin Giao Thức\par}

    \vspace{1cm}
    {\Large English Learning Application\par}

    \vspace{2cm}

    \begin{tikzpicture}[scale=0.8]
        % Client-Server Architecture
        \node[rectangle, draw=clientcolor, fill=clientcolor!20, minimum width=4cm, minimum height=2cm, rounded corners] (client) at (0,0) {\Large Client};
        \node[rectangle, draw=servercolor, fill=servercolor!20, minimum width=4cm, minimum height=2cm, rounded corners] (server) at (8,0) {\Large Server};

        % Arrows
        \draw[->, thick, clientcolor] (client.east) -- node[above, font=\small] {Request} (server.west);
        \draw[->, thick, servercolor] (server.west) ++(0,-0.3) -- node[below, font=\small] {Response} ++(-4,0);
    \end{tikzpicture}

    \vspace{2cm}

    {\large Phiên bản: 1.0\par}
    {\large Ngày: \today\par}

    \vfill

    {\large Tài liệu mô tả chi tiết các quy trình\\và bản tin sử dụng cho từng chức năng\par}
\end{titlepage}

\tableofcontents
\newpage

%====================================================================
\section{Tổng Quan}
%====================================================================

\subsection{Giới Thiệu}

Tài liệu này mô tả chi tiết các quy trình (flows) của từng chức năng trong ứng dụng English Learning, bao gồm:
\begin{enumerate}
    \item \textbf{Đăng ký/Đăng nhập} - Quản lý tài khoản và phiên làm việc
    \item \textbf{Học bài} - Xem và học các bài học
    \item \textbf{Làm bài kiểm tra} - Tham gia các bài test
    \item \textbf{Làm bài tập} - Hoàn thành bài tập và nhận phản hồi
    \item \textbf{Chat} - Giao tiếp với người dùng khác
    \item \textbf{Chơi game} - Tham gia các trò chơi học tiếng Anh
\end{enumerate}

\subsection{Quy Ước Bản Tin}

\begin{table}[H]
\centering
\begin{tabular}{|l|l|l|}
\hline
\textbf{Loại} & \textbf{Hướng} & \textbf{Mô tả} \\
\hline
REQUEST & Client $\rightarrow$ Server & Yêu cầu từ client \\
RESPONSE & Server $\rightarrow$ Client & Phản hồi từ server \\
NOTIFICATION & Server $\rightarrow$ Client & Thông báo push (không cần request) \\
\hline
\end{tabular}
\caption{Phân loại bản tin}
\end{table}

\subsection{Cấu Trúc Bản Tin Chung}

\begin{lstlisting}[caption={Cấu trúc JSON chung}]
{
    "messageType": "MESSAGE_TYPE",
    "sessionToken": "token_string",    // Bat buoc sau khi dang nhap
    "timestamp": 1704067200000,        // Unix timestamp (ms)
    "payload": {
        // Du lieu cu the cho tung loai ban tin
    }
}
\end{lstlisting}

\newpage
%====================================================================
\section{Đăng Ký / Đăng Nhập (Authentication)}
%====================================================================

\subsection{Mô Tả Chức Năng}

Chức năng xác thực cho phép người dùng:
\begin{itemize}
    \item Đăng ký tài khoản mới
    \item Đăng nhập vào hệ thống
    \item Nhận token phiên làm việc
    \item Phân quyền theo vai trò (Student/Teacher/Admin)
\end{itemize}

\subsection{Quy Trình Đăng Ký}

\begin{figure}[H]
\centering
\begin{tikzpicture}[node distance=1.5cm, scale=0.85, transform shape]
    % Nodes
    \node (start) [startstop] {Bắt đầu};
    \node (input) [io, below of=start] {Nhập thông tin};
    \node (validate) [process, below of=input] {Validate dữ liệu};
    \node (check) [decision, below of=validate, yshift=-0.5cm] {Hợp lệ?};
    \node (send) [process, below of=check, yshift=-0.5cm] {Gửi REGISTER\_REQUEST};
    \node (wait) [process, below of=send] {Chờ response};
    \node (checkResp) [decision, below of=wait, yshift=-0.5cm] {Success?};
    \node (success) [startstop, below of=checkResp, yshift=-0.5cm] {Đăng ký thành công};
    \node (error) [process, right of=check, xshift=3cm] {Hiển thị lỗi};
    \node (retry) [process, right of=checkResp, xshift=3cm] {Hiển thị lỗi};

    % Arrows
    \draw [arrow] (start) -- (input);
    \draw [arrow] (input) -- (validate);
    \draw [arrow] (validate) -- (check);
    \draw [arrow] (check) -- node[left] {Có} (send);
    \draw [arrow] (check) -- node[above] {Không} (error);
    \draw [arrow] (error) |- (input);
    \draw [arrow] (send) -- (wait);
    \draw [arrow] (wait) -- (checkResp);
    \draw [arrow] (checkResp) -- node[left] {Có} (success);
    \draw [arrow] (checkResp) -- node[above] {Không} (retry);
    \draw [arrow] (retry) |- (input);
\end{tikzpicture}
\caption{Flowchart quy trình đăng ký}
\end{figure}

\subsection{Quy Trình Đăng Nhập}

\begin{figure}[H]
\centering
\begin{tikzpicture}[node distance=1.5cm, scale=0.85, transform shape]
    % Nodes
    \node (start) [startstop] {Bắt đầu};
    \node (input) [io, below of=start] {Nhập Email/Password};
    \node (validate) [process, below of=input] {Validate định dạng};
    \node (check) [decision, below of=validate, yshift=-0.5cm] {Hợp lệ?};
    \node (connect) [process, below of=check, yshift=-0.5cm] {Kết nối Server};
    \node (send) [process, below of=connect] {Gửi LOGIN\_REQUEST};
    \node (wait) [process, below of=send] {Chờ LOGIN\_RESPONSE};
    \node (checkResp) [decision, below of=wait, yshift=-0.5cm] {Success?};
    \node (store) [process, below of=checkResp, yshift=-0.5cm] {Lưu sessionToken};
    \node (menu) [process, below of=store] {Hiển thị menu theo role};
    \node (end) [startstop, below of=menu] {Hoàn thành};
    \node (error) [process, right of=check, xshift=3cm] {Hiển thị lỗi};
    \node (retry) [process, right of=checkResp, xshift=3cm] {Hiển thị lỗi};

    % Arrows
    \draw [arrow] (start) -- (input);
    \draw [arrow] (input) -- (validate);
    \draw [arrow] (validate) -- (check);
    \draw [arrow] (check) -- node[left] {Có} (connect);
    \draw [arrow] (check) -- node[above] {Không} (error);
    \draw [arrow] (error) |- (input);
    \draw [arrow] (connect) -- (send);
    \draw [arrow] (send) -- (wait);
    \draw [arrow] (wait) -- (checkResp);
    \draw [arrow] (checkResp) -- node[left] {Có} (store);
    \draw [arrow] (checkResp) -- node[above] {Không} (retry);
    \draw [arrow] (retry) |- (input);
    \draw [arrow] (store) -- (menu);
    \draw [arrow] (menu) -- (end);
\end{tikzpicture}
\caption{Flowchart quy trình đăng nhập}
\end{figure}

\subsection{Sequence Diagram - Đăng Ký}

\begin{figure}[H]
\centering
\begin{tikzpicture}[scale=0.9, transform shape]
    % Lifelines
    \node[client] (client) at (0,0) {Client};
    \node[server] (server) at (8,0) {Server};

    \draw[dashed] (0,-0.5) -- (0,-8);
    \draw[dashed] (8,-0.5) -- (8,-8);

    % Messages
    \draw[->, thick, clientcolor] (0,-1.5) -- node[above, font=\scriptsize] {REGISTER\_REQUEST} (8,-1.5);
    \draw[<-, thick, servercolor] (0,-3) -- node[above, font=\scriptsize] {REGISTER\_RESPONSE (success)} (8,-3);

    % Activation boxes
    \fill[clientcolor!30] (-0.15,-1.3) rectangle (0.15,-1.7);
    \fill[servercolor!30] (7.85,-1.5) rectangle (8.15,-2.8);

    % Notes
    \node[font=\scriptsize, align=left] at (4,-2.2) {Server validate \& tạo user};

    % Alternative flow
    \draw[dashed] (-1,-4) rectangle (9,-6);
    \node[font=\scriptsize] at (0,-4.3) {alt [Email đã tồn tại]};
    \draw[<-, thick, red] (0,-5) -- node[above, font=\scriptsize] {REGISTER\_RESPONSE (error)} (8,-5);
\end{tikzpicture}
\caption{Sequence diagram đăng ký}
\end{figure}

\subsection{Sequence Diagram - Đăng Nhập}

\begin{figure}[H]
\centering
\begin{tikzpicture}[scale=0.9, transform shape]
    % Lifelines
    \node[client] (client) at (0,0) {Client};
    \node[server] (server) at (8,0) {Server};

    \draw[dashed] (0,-0.5) -- (0,-10);
    \draw[dashed] (8,-0.5) -- (8,-10);

    % Connect
    \draw[->, thick] (0,-1.5) -- node[above, font=\scriptsize] {TCP Connect} (8,-1.5);
    \draw[<-, thick] (0,-2) -- node[above, font=\scriptsize] {Connection OK} (8,-2);

    % Login
    \draw[->, thick, clientcolor] (0,-3) -- node[above, font=\scriptsize] {LOGIN\_REQUEST} (8,-3);

    % Server processing
    \fill[servercolor!30] (7.85,-3) rectangle (8.15,-5.5);
    \node[font=\scriptsize, align=left, anchor=west] at (8.3,-4) {Validate credentials\\Create session\\Generate token};

    \draw[<-, thick, servercolor] (0,-5.5) -- node[above, font=\scriptsize] {LOGIN\_RESPONSE} (8,-5.5);

    % Client processing
    \fill[clientcolor!30] (-0.15,-5.5) rectangle (0.15,-7);
    \node[font=\scriptsize, align=right, anchor=east] at (-0.3,-6.25) {Store token\\Parse role\\Show menu};

    % Alternative
    \draw[dashed] (-1,-7.5) rectangle (9,-9.5);
    \node[font=\scriptsize] at (0,-7.8) {alt [Invalid credentials]};
    \draw[<-, thick, red] (0,-8.5) -- node[above, font=\scriptsize] {LOGIN\_RESPONSE (error)} (8,-8.5);
\end{tikzpicture}
\caption{Sequence diagram đăng nhập}
\end{figure}

\subsection{Danh Sách Bản Tin}

\begin{longtable}{|p{5cm}|p{2.5cm}|p{6cm}|}
\hline
\textbf{Message Type} & \textbf{Hướng} & \textbf{Mô tả} \\
\hline
\endhead
REGISTER\_REQUEST & C $\rightarrow$ S & Yêu cầu đăng ký tài khoản mới \\
\hline
REGISTER\_RESPONSE & S $\rightarrow$ C & Kết quả đăng ký (success/error) \\
\hline
LOGIN\_REQUEST & C $\rightarrow$ S & Yêu cầu đăng nhập \\
\hline
LOGIN\_RESPONSE & S $\rightarrow$ C & Kết quả đăng nhập + sessionToken + role \\
\hline
SET\_LEVEL\_REQUEST & C $\rightarrow$ S & Cập nhật level người dùng \\
\hline
SET\_LEVEL\_RESPONSE & S $\rightarrow$ C & Kết quả cập nhật level \\
\hline
\caption{Danh sách bản tin Authentication}
\end{longtable}

\subsection{Chi Tiết Bản Tin}

\subsubsection{REGISTER\_REQUEST}

\begin{lstlisting}[caption={REGISTER\_REQUEST}]
{
    "messageType": "REGISTER_REQUEST",
    "payload": {
        "fullname": "Nguyen Van A",
        "email": "nguyenvana@example.com",
        "password": "password123",
        "confirmPassword": "password123"
    }
}
\end{lstlisting}

\begin{table}[H]
\centering
\begin{tabular}{|l|l|l|p{5cm}|}
\hline
\textbf{Trường} & \textbf{Kiểu} & \textbf{Bắt buộc} & \textbf{Mô tả} \\
\hline
fullname & string & Có & Họ tên người dùng \\
email & string & Có & Email (unique) \\
password & string & Có & Mật khẩu ($\geq$ 6 ký tự) \\
confirmPassword & string & Có & Xác nhận mật khẩu \\
\hline
\end{tabular}
\end{table}

\subsubsection{REGISTER\_RESPONSE}

\begin{lstlisting}[caption={REGISTER\_RESPONSE - Success}]
{
    "messageType": "REGISTER_RESPONSE",
    "timestamp": 1704067200000,
    "payload": {
        "status": "success",
        "message": "Registration successful",
        "data": {
            "userId": "user_abc123"
        }
    }
}
\end{lstlisting}

\subsubsection{LOGIN\_REQUEST}

\begin{lstlisting}[caption={LOGIN\_REQUEST}]
{
    "messageType": "LOGIN_REQUEST",
    "payload": {
        "email": "student@example.com",
        "password": "password123"
    }
}
\end{lstlisting}

\subsubsection{LOGIN\_RESPONSE}

\begin{lstlisting}[caption={LOGIN\_RESPONSE - Success}]
{
    "messageType": "LOGIN_RESPONSE",
    "timestamp": 1704067200000,
    "payload": {
        "status": "success",
        "message": "Login successfully",
        "data": {
            "userId": "user_001",
            "fullname": "Nguyen Van A",
            "email": "student@example.com",
            "level": "beginner",
            "role": "student",
            "sessionToken": "token_xyz789",
            "expiresAt": 1704070800000
        }
    }
}
\end{lstlisting}

\begin{table}[H]
\centering
\begin{tabular}{|l|l|p{6cm}|}
\hline
\textbf{Trường} & \textbf{Kiểu} & \textbf{Mô tả} \\
\hline
userId & string & ID người dùng \\
fullname & string & Họ tên \\
email & string & Email \\
level & string & beginner/intermediate/advanced \\
role & string & student/teacher/admin \\
sessionToken & string & Token phiên làm việc \\
expiresAt & int64 & Thời điểm hết hạn (Unix ms) \\
\hline
\end{tabular}
\end{table}

\newpage
%====================================================================
\section{Học Bài (Lessons)}
%====================================================================

\subsection{Mô Tả Chức Năng}

Chức năng học bài cho phép:
\begin{itemize}
    \item Xem danh sách bài học theo level và topic
    \item Xem chi tiết nội dung bài học
    \item Xem video và nghe audio bài học
\end{itemize}

\subsection{Quy Trình Học Bài}

\begin{figure}[H]
\centering
\begin{tikzpicture}[node distance=1.5cm, scale=0.85, transform shape]
    % Nodes
    \node (start) [startstop] {Chọn "Học bài"};
    \node (send1) [process, below of=start] {Gửi GET\_LESSONS\_REQUEST};
    \node (wait1) [process, below of=send1] {Nhận GET\_LESSONS\_RESPONSE};
    \node (display) [io, below of=wait1] {Hiển thị danh sách bài};
    \node (select) [process, below of=display] {Chọn bài học};
    \node (send2) [process, below of=select] {Gửi GET\_LESSON\_DETAIL\_REQUEST};
    \node (wait2) [process, below of=send2] {Nhận GET\_LESSON\_DETAIL\_RESPONSE};
    \node (show) [io, below of=wait2] {Hiển thị nội dung};
    \node (media) [decision, below of=show, yshift=-0.5cm] {Có media?};
    \node (play) [process, below of=media, yshift=-0.5cm] {Phát video/audio};
    \node (end) [startstop, below of=play] {Hoàn thành};
    \node (skip) [process, right of=media, xshift=3cm] {Bỏ qua};

    % Arrows
    \draw [arrow] (start) -- (send1);
    \draw [arrow] (send1) -- (wait1);
    \draw [arrow] (wait1) -- (display);
    \draw [arrow] (display) -- (select);
    \draw [arrow] (select) -- (send2);
    \draw [arrow] (send2) -- (wait2);
    \draw [arrow] (wait2) -- (show);
    \draw [arrow] (show) -- (media);
    \draw [arrow] (media) -- node[left] {Có} (play);
    \draw [arrow] (media) -- node[above] {Không} (skip);
    \draw [arrow] (play) -- (end);
    \draw [arrow] (skip) |- (end);
\end{tikzpicture}
\caption{Flowchart quy trình học bài}
\end{figure}

\subsection{Sequence Diagram}

\begin{figure}[H]
\centering
\begin{tikzpicture}[scale=0.85, transform shape]
    % Lifelines
    \node[client] (client) at (0,0) {Client};
    \node[server] (server) at (8,0) {Server};

    \draw[dashed] (0,-0.5) -- (0,-12);
    \draw[dashed] (8,-0.5) -- (8,-12);

    % Get lessons list
    \draw[->, thick, clientcolor] (0,-1.5) -- node[above, font=\scriptsize] {GET\_LESSONS\_REQUEST} (8,-1.5);
    \fill[servercolor!30] (7.85,-1.5) rectangle (8.15,-2.5);
    \draw[<-, thick, servercolor] (0,-3) -- node[above, font=\scriptsize] {GET\_LESSONS\_RESPONSE} (8,-3);

    % User selects
    \node[font=\scriptsize, align=center] at (0,-4) {User chọn\\bài học};

    % Get lesson detail
    \draw[->, thick, clientcolor] (0,-5) -- node[above, font=\scriptsize] {GET\_LESSON\_DETAIL\_REQUEST} (8,-5);
    \fill[servercolor!30] (7.85,-5) rectangle (8.15,-6);
    \draw[<-, thick, servercolor] (0,-6.5) -- node[above, font=\scriptsize] {GET\_LESSON\_DETAIL\_RESPONSE} (8,-6.5);

    % Display content
    \fill[clientcolor!30] (-0.15,-6.5) rectangle (0.15,-8);
    \node[font=\scriptsize, align=right, anchor=east] at (-0.3,-7.25) {Parse content\\Display text\\Load media};

    % Optional media
    \draw[dashed] (-1,-8.5) rectangle (9,-11);
    \node[font=\scriptsize] at (0,-8.8) {opt [Có video/audio]};
    \node[font=\scriptsize, align=center] at (0,-10) {Phát media\\từ URL};
\end{tikzpicture}
\caption{Sequence diagram học bài}
\end{figure}

\subsection{Danh Sách Bản Tin}

\begin{longtable}{|p{5.5cm}|p{2.5cm}|p{5.5cm}|}
\hline
\textbf{Message Type} & \textbf{Hướng} & \textbf{Mô tả} \\
\hline
\endhead
GET\_LESSONS\_REQUEST & C $\rightarrow$ S & Lấy danh sách bài học \\
\hline
GET\_LESSONS\_RESPONSE & S $\rightarrow$ C & Danh sách bài học \\
\hline
GET\_LESSON\_DETAIL\_REQUEST & C $\rightarrow$ S & Lấy chi tiết bài học \\
\hline
GET\_LESSON\_DETAIL\_RESPONSE & S $\rightarrow$ C & Nội dung chi tiết bài học \\
\hline
\caption{Danh sách bản tin Lesson}
\end{longtable}

\subsection{Chi Tiết Bản Tin}

\subsubsection{GET\_LESSONS\_REQUEST}

\begin{lstlisting}[caption={GET\_LESSONS\_REQUEST}]
{
    "messageType": "GET_LESSONS_REQUEST",
    "sessionToken": "token_xyz789",
    "payload": {
        "level": "beginner",
        "topic": "grammar"
    }
}
\end{lstlisting}

\subsubsection{GET\_LESSONS\_RESPONSE}

\begin{lstlisting}[caption={GET\_LESSONS\_RESPONSE}]
{
    "messageType": "GET_LESSONS_RESPONSE",
    "payload": {
        "status": "success",
        "data": {
            "total": 2,
            "lessons": [
                {
                    "lessonId": "lesson_001",
                    "title": "Basic Grammar",
                    "topic": "grammar",
                    "level": "beginner",
                    "description": "Learn basic English grammar"
                }
            ]
        }
    }
}
\end{lstlisting}

\subsubsection{GET\_LESSON\_DETAIL\_REQUEST}

\begin{lstlisting}[caption={GET\_LESSON\_DETAIL\_REQUEST}]
{
    "messageType": "GET_LESSON_DETAIL_REQUEST",
    "sessionToken": "token_xyz789",
    "payload": {
        "lessonId": "lesson_001"
    }
}
\end{lstlisting}

\subsubsection{GET\_LESSON\_DETAIL\_RESPONSE}

\begin{lstlisting}[caption={GET\_LESSON\_DETAIL\_RESPONSE}]
{
    "messageType": "GET_LESSON_DETAIL_RESPONSE",
    "payload": {
        "status": "success",
        "data": {
            "lessonId": "lesson_001",
            "title": "Basic Grammar",
            "description": "Learn basic English grammar",
            "level": "beginner",
            "topic": "grammar",
            "textContent": "# Introduction to Grammar...",
            "videoUrl": "https://example.com/video.mp4",
            "audioUrl": "https://example.com/audio.mp3"
        }
    }
}
\end{lstlisting}

\newpage
%====================================================================
\section{Làm Bài Kiểm Tra (Tests)}
%====================================================================

\subsection{Mô Tả Chức Năng}

Chức năng làm bài kiểm tra cho phép:
\begin{itemize}
    \item Lấy bài test theo level
    \item Làm các loại câu hỏi: multiple choice, fill blank, sentence order
    \item Nộp bài và nhận kết quả chấm điểm
\end{itemize}

\subsection{Quy Trình Làm Bài Kiểm Tra}

\begin{figure}[H]
\centering
\begin{tikzpicture}[node distance=1.3cm, scale=0.8, transform shape]
    % Nodes
    \node (start) [startstop] {Chọn "Làm bài test"};
    \node (send1) [process, below of=start] {Gửi GET\_TEST\_REQUEST};
    \node (wait1) [process, below of=send1] {Nhận GET\_TEST\_RESPONSE};
    \node (display) [io, below of=wait1] {Hiển thị câu hỏi};
    \node (timer) [process, below of=display] {Bắt đầu đếm giờ};
    \node (answer) [io, below of=timer] {User trả lời};
    \node (check) [decision, below of=answer, yshift=-0.3cm] {Còn câu hỏi?};
    \node (next) [process, right of=check, xshift=3cm] {Câu tiếp theo};
    \node (confirm) [decision, below of=check, yshift=-0.5cm] {Nộp bài?};
    \node (send2) [process, below of=confirm, yshift=-0.3cm] {Gửi SUBMIT\_TEST\_REQUEST};
    \node (wait2) [process, below of=send2] {Nhận SUBMIT\_TEST\_RESPONSE};
    \node (result) [io, below of=wait2] {Hiển thị kết quả};
    \node (end) [startstop, below of=result] {Hoàn thành};
    \node (back) [process, right of=confirm, xshift=3cm] {Quay lại sửa};

    % Arrows
    \draw [arrow] (start) -- (send1);
    \draw [arrow] (send1) -- (wait1);
    \draw [arrow] (wait1) -- (display);
    \draw [arrow] (display) -- (timer);
    \draw [arrow] (timer) -- (answer);
    \draw [arrow] (answer) -- (check);
    \draw [arrow] (check) -- node[above] {Có} (next);
    \draw [arrow] (next) |- (answer);
    \draw [arrow] (check) -- node[left] {Không} (confirm);
    \draw [arrow] (confirm) -- node[left] {Có} (send2);
    \draw [arrow] (confirm) -- node[above] {Không} (back);
    \draw [arrow] (back) |- (answer);
    \draw [arrow] (send2) -- (wait2);
    \draw [arrow] (wait2) -- (result);
    \draw [arrow] (result) -- (end);
\end{tikzpicture}
\caption{Flowchart quy trình làm bài kiểm tra}
\end{figure}

\subsection{Sequence Diagram}

\begin{figure}[H]
\centering
\begin{tikzpicture}[scale=0.85, transform shape]
    % Lifelines
    \node[client] (client) at (0,0) {Client};
    \node[server] (server) at (8,0) {Server};

    \draw[dashed] (0,-0.5) -- (0,-14);
    \draw[dashed] (8,-0.5) -- (8,-14);

    % Get test
    \draw[->, thick, clientcolor] (0,-1.5) -- node[above, font=\scriptsize] {GET\_TEST\_REQUEST} (8,-1.5);
    \fill[servercolor!30] (7.85,-1.5) rectangle (8.15,-2.5);
    \draw[<-, thick, servercolor] (0,-3) -- node[above, font=\scriptsize] {GET\_TEST\_RESPONSE} (8,-3);

    % User doing test
    \draw[dashed] (-1,-4) rectangle (1,-7);
    \node[font=\scriptsize] at (0,-4.3) {loop [Làm bài]};
    \node[font=\scriptsize, align=center] at (0,-5.5) {Hiển thị câu hỏi\\User trả lời\\Lưu đáp án};

    % Submit
    \draw[->, thick, clientcolor] (0,-8) -- node[above, font=\scriptsize] {SUBMIT\_TEST\_REQUEST} (8,-8);

    % Server grading
    \fill[servercolor!30] (7.85,-8) rectangle (8.15,-10);
    \node[font=\scriptsize, align=left, anchor=west] at (8.3,-9) {Chấm điểm\\So sánh đáp án\\Tính tổng điểm};

    \draw[<-, thick, servercolor] (0,-10.5) -- node[above, font=\scriptsize] {SUBMIT\_TEST\_RESPONSE} (8,-10.5);

    % Display result
    \fill[clientcolor!30] (-0.15,-10.5) rectangle (0.15,-12);
    \node[font=\scriptsize, align=right, anchor=east] at (-0.3,-11.25) {Hiển thị điểm\\Chi tiết đáp án\\Grade/Pass};
\end{tikzpicture}
\caption{Sequence diagram làm bài kiểm tra}
\end{figure}

\subsection{Danh Sách Bản Tin}

\begin{longtable}{|p{5cm}|p{2.5cm}|p{6cm}|}
\hline
\textbf{Message Type} & \textbf{Hướng} & \textbf{Mô tả} \\
\hline
\endhead
GET\_TEST\_REQUEST & C $\rightarrow$ S & Lấy bài test theo level \\
\hline
GET\_TEST\_RESPONSE & S $\rightarrow$ C & Dữ liệu bài test + câu hỏi \\
\hline
SUBMIT\_TEST\_REQUEST & C $\rightarrow$ S & Nộp bài test \\
\hline
SUBMIT\_TEST\_RESPONSE & S $\rightarrow$ C & Kết quả chấm điểm \\
\hline
\caption{Danh sách bản tin Test}
\end{longtable}

\subsection{Chi Tiết Bản Tin}

\subsubsection{GET\_TEST\_REQUEST}

\begin{lstlisting}[caption={GET\_TEST\_REQUEST}]
{
    "messageType": "GET_TEST_REQUEST",
    "sessionToken": "token_xyz789",
    "payload": {
        "level": "beginner"
    }
}
\end{lstlisting}

\subsubsection{GET\_TEST\_RESPONSE}

\begin{lstlisting}[caption={GET\_TEST\_RESPONSE}]
{
    "messageType": "GET_TEST_RESPONSE",
    "payload": {
        "status": "success",
        "data": {
            "testId": "test_001",
            "title": "Beginner English Test",
            "timeLimit": 1800,
            "totalPoints": 100,
            "questions": [
                {
                    "questionId": "q1",
                    "type": "multiple_choice",
                    "question": "What is 'apple' in Vietnamese?",
                    "points": 10,
                    "options": [
                        {"id": "a", "text": "Qua cam"},
                        {"id": "b", "text": "Qua tao"},
                        {"id": "c", "text": "Qua chuoi"},
                        {"id": "d", "text": "Qua nho"}
                    ]
                },
                {
                    "questionId": "q2",
                    "type": "fill_blank",
                    "question": "She ___ a student.",
                    "points": 10
                },
                {
                    "questionId": "q3",
                    "type": "sentence_order",
                    "question": "Arrange the words",
                    "points": 15,
                    "words": ["is", "a", "This", "book"]
                }
            ]
        }
    }
}
\end{lstlisting}

\subsubsection{SUBMIT\_TEST\_REQUEST}

\begin{lstlisting}[caption={SUBMIT\_TEST\_REQUEST}]
{
    "messageType": "SUBMIT_TEST_REQUEST",
    "sessionToken": "token_xyz789",
    "payload": {
        "testId": "test_001",
        "answers": [
            {"questionId": "q1", "answer": "b"},
            {"questionId": "q2", "answer": "is"},
            {"questionId": "q3", "answer": "This is a book"}
        ]
    }
}
\end{lstlisting}

\subsubsection{SUBMIT\_TEST\_RESPONSE}

\begin{lstlisting}[caption={SUBMIT\_TEST\_RESPONSE}]
{
    "messageType": "SUBMIT_TEST_RESPONSE",
    "payload": {
        "status": "success",
        "data": {
            "score": 85,
            "totalPoints": 100,
            "percentage": "85%",
            "passed": true,
            "grade": "B",
            "correctAnswers": 8,
            "wrongAnswers": 2,
            "results": [
                {
                    "questionId": "q1",
                    "correct": true,
                    "userAnswer": "b",
                    "correctAnswer": "b"
                }
            ]
        }
    }
}
\end{lstlisting}

\newpage
%====================================================================
\section{Làm Bài Tập (Exercises)}
%====================================================================

\subsection{Mô Tả Chức Năng}

Chức năng làm bài tập bao gồm:

\textbf{Dành cho Student:}
\begin{itemize}
    \item Xem danh sách bài tập
    \item Làm bài tập (writing, speaking, listening)
    \item Lưu bản nháp
    \item Nộp bài tập
    \item Xem phản hồi từ giáo viên
\end{itemize}

\textbf{Dành cho Teacher:}
\begin{itemize}
    \item Xem danh sách bài nộp chờ chấm
    \item Xem chi tiết bài nộp
    \item Chấm điểm và gửi phản hồi
\end{itemize}

\subsection{Quy Trình Student - Làm Bài Tập}

\begin{figure}[H]
\centering
\begin{tikzpicture}[node distance=1.3cm, scale=0.75, transform shape]
    % Nodes
    \node (start) [startstop] {Chọn "Bài tập"};
    \node (send1) [process, below of=start] {GET\_EXERCISE\_LIST\_REQUEST};
    \node (wait1) [process, below of=send1] {GET\_EXERCISE\_LIST\_RESPONSE};
    \node (display) [io, below of=wait1] {Hiển thị danh sách};
    \node (select) [process, below of=display] {Chọn bài tập};
    \node (send2) [process, below of=select] {GET\_EXERCISE\_REQUEST};
    \node (wait2) [process, below of=send2] {GET\_EXERCISE\_RESPONSE};
    \node (do) [io, below of=wait2] {Làm bài};
    \node (action) [decision, below of=do, yshift=-0.5cm] {Hành động?};
    \node (save) [process, below of=action, yshift=-0.5cm, xshift=-3cm] {SAVE\_DRAFT\_REQUEST};
    \node (submit) [process, below of=action, yshift=-0.5cm, xshift=3cm] {SUBMIT\_EXERCISE\_REQUEST};
    \node (end) [startstop, below of=action, yshift=-2.5cm] {Hoàn thành};

    % Arrows
    \draw [arrow] (start) -- (send1);
    \draw [arrow] (send1) -- (wait1);
    \draw [arrow] (wait1) -- (display);
    \draw [arrow] (display) -- (select);
    \draw [arrow] (select) -- (send2);
    \draw [arrow] (send2) -- (wait2);
    \draw [arrow] (wait2) -- (do);
    \draw [arrow] (do) -- (action);
    \draw [arrow] (action) -| node[above left, pos=0.3] {Lưu nháp} (save);
    \draw [arrow] (action) -| node[above right, pos=0.3] {Nộp bài} (submit);
    \draw [arrow] (save) |- (end);
    \draw [arrow] (submit) |- (end);
\end{tikzpicture}
\caption{Flowchart quy trình làm bài tập (Student)}
\end{figure}

\subsection{Quy Trình Teacher - Chấm Bài}

\begin{figure}[H]
\centering
\begin{tikzpicture}[node distance=1.3cm, scale=0.75, transform shape]
    % Nodes
    \node (start) [startstop] {Chọn "Chấm bài"};
    \node (send1) [process, below of=start] {GET\_PENDING\_REVIEWS\_REQUEST};
    \node (wait1) [process, below of=send1] {GET\_PENDING\_REVIEWS\_RESPONSE};
    \node (display) [io, below of=wait1] {Hiển thị bài chờ chấm};
    \node (select) [process, below of=display] {Chọn bài nộp};
    \node (send2) [process, below of=select] {GET\_SUBMISSION\_DETAIL\_REQUEST};
    \node (wait2) [process, below of=send2] {GET\_SUBMISSION\_DETAIL\_RESPONSE};
    \node (review) [io, below of=wait2] {Xem \& chấm bài};
    \node (send3) [process, below of=review] {REVIEW\_EXERCISE\_REQUEST};
    \node (wait3) [process, below of=send3] {REVIEW\_EXERCISE\_RESPONSE};
    \node (end) [startstop, below of=wait3] {Hoàn thành};

    % Arrows
    \draw [arrow] (start) -- (send1);
    \draw [arrow] (send1) -- (wait1);
    \draw [arrow] (wait1) -- (display);
    \draw [arrow] (display) -- (select);
    \draw [arrow] (select) -- (send2);
    \draw [arrow] (send2) -- (wait2);
    \draw [arrow] (wait2) -- (review);
    \draw [arrow] (review) -- (send3);
    \draw [arrow] (send3) -- (wait3);
    \draw [arrow] (wait3) -- (end);
\end{tikzpicture}
\caption{Flowchart quy trình chấm bài (Teacher)}
\end{figure}

\subsection{Sequence Diagram - Làm Bài Tập}

\begin{figure}[H]
\centering
\begin{tikzpicture}[scale=0.75, transform shape]
    % Lifelines
    \node[client] (student) at (0,0) {Student};
    \node[server] (server) at (6,0) {Server};
    \node[client] (teacher) at (12,0) {Teacher};

    \draw[dashed] (0,-0.5) -- (0,-16);
    \draw[dashed] (6,-0.5) -- (6,-16);
    \draw[dashed] (12,-0.5) -- (12,-16);

    % Student gets exercise
    \draw[->, thick, clientcolor] (0,-1.5) -- node[above, font=\scriptsize] {GET\_EXERCISE\_LIST\_REQUEST} (6,-1.5);
    \draw[<-, thick, servercolor] (0,-2.5) -- node[above, font=\scriptsize] {GET\_EXERCISE\_LIST\_RESPONSE} (6,-2.5);

    \draw[->, thick, clientcolor] (0,-3.5) -- node[above, font=\scriptsize] {GET\_EXERCISE\_REQUEST} (6,-3.5);
    \draw[<-, thick, servercolor] (0,-4.5) -- node[above, font=\scriptsize] {GET\_EXERCISE\_RESPONSE} (6,-4.5);

    % Student saves draft
    \draw[dashed] (-1,-5.5) rectangle (7,-7.5);
    \node[font=\scriptsize] at (0,-5.8) {opt [Lưu nháp]};
    \draw[->, thick, clientcolor] (0,-6.5) -- node[above, font=\scriptsize] {SAVE\_DRAFT\_REQUEST} (6,-6.5);
    \draw[<-, thick, servercolor] (0,-7.2) -- node[above, font=\scriptsize] {SAVE\_DRAFT\_RESPONSE} (6,-7.2);

    % Student submits
    \draw[->, thick, clientcolor] (0,-8.5) -- node[above, font=\scriptsize] {SUBMIT\_EXERCISE\_REQUEST} (6,-8.5);
    \draw[<-, thick, servercolor] (0,-9.5) -- node[above, font=\scriptsize] {SUBMIT\_EXERCISE\_RESPONSE} (6,-9.5);

    % Notification to teacher
    \draw[->, thick, notifycolor] (6,-10) -- node[above, font=\scriptsize] {NEW\_SUBMISSION\_NOTIFICATION} (12,-10);

    % Teacher reviews
    \draw[->, thick, clientcolor] (12,-11) -- node[above, font=\scriptsize] {GET\_SUBMISSION\_DETAIL\_REQUEST} (6,-11);
    \draw[<-, thick, servercolor] (12,-12) -- node[above, font=\scriptsize] {GET\_SUBMISSION\_DETAIL\_RESPONSE} (6,-12);

    \draw[->, thick, clientcolor] (12,-13) -- node[above, font=\scriptsize] {REVIEW\_EXERCISE\_REQUEST} (6,-13);
    \draw[<-, thick, servercolor] (12,-14) -- node[above, font=\scriptsize] {REVIEW\_EXERCISE\_RESPONSE} (6,-14);

    % Notification to student
    \draw[->, thick, notifycolor] (6,-15) -- node[above, font=\scriptsize] {EXERCISE\_FEEDBACK\_NOTIFICATION} (0,-15);
\end{tikzpicture}
\caption{Sequence diagram làm bài tập và chấm bài}
\end{figure}

\subsection{Danh Sách Bản Tin}

\begin{longtable}{|p{5.5cm}|p{2.5cm}|p{5.5cm}|}
\hline
\textbf{Message Type} & \textbf{Hướng} & \textbf{Mô tả} \\
\hline
\endhead
\multicolumn{3}{|c|}{\textbf{Student Workflow}} \\
\hline
GET\_EXERCISE\_LIST\_REQUEST & C $\rightarrow$ S & Lấy danh sách bài tập \\
\hline
GET\_EXERCISE\_LIST\_RESPONSE & S $\rightarrow$ C & Danh sách bài tập \\
\hline
GET\_EXERCISE\_REQUEST & C $\rightarrow$ S & Lấy chi tiết bài tập \\
\hline
GET\_EXERCISE\_RESPONSE & S $\rightarrow$ C & Nội dung bài tập \\
\hline
SAVE\_DRAFT\_REQUEST & C $\rightarrow$ S & Lưu bản nháp \\
\hline
SAVE\_DRAFT\_RESPONSE & S $\rightarrow$ C & Xác nhận lưu \\
\hline
SUBMIT\_EXERCISE\_REQUEST & C $\rightarrow$ S & Nộp bài tập \\
\hline
SUBMIT\_EXERCISE\_RESPONSE & S $\rightarrow$ C & Xác nhận nộp \\
\hline
GET\_USER\_SUBMISSIONS\_REQUEST & C $\rightarrow$ S & Xem các bài đã nộp \\
\hline
GET\_USER\_SUBMISSIONS\_RESPONSE & S $\rightarrow$ C & Danh sách bài nộp \\
\hline
GET\_FEEDBACK\_REQUEST & C $\rightarrow$ S & Xem phản hồi \\
\hline
GET\_FEEDBACK\_RESPONSE & S $\rightarrow$ C & Chi tiết phản hồi \\
\hline
GET\_MY\_DRAFTS\_REQUEST & C $\rightarrow$ S & Xem bản nháp \\
\hline
GET\_MY\_DRAFTS\_RESPONSE & S $\rightarrow$ C & Danh sách bản nháp \\
\hline
\multicolumn{3}{|c|}{\textbf{Teacher Workflow}} \\
\hline
GET\_PENDING\_REVIEWS\_REQUEST & C $\rightarrow$ S & Lấy bài chờ chấm \\
\hline
GET\_PENDING\_REVIEWS\_RESPONSE & S $\rightarrow$ C & Danh sách bài chờ \\
\hline
GET\_SUBMISSION\_DETAIL\_REQUEST & C $\rightarrow$ S & Chi tiết bài nộp \\
\hline
GET\_SUBMISSION\_DETAIL\_RESPONSE & S $\rightarrow$ C & Nội dung bài nộp \\
\hline
REVIEW\_EXERCISE\_REQUEST & C $\rightarrow$ S & Chấm điểm + phản hồi \\
\hline
REVIEW\_EXERCISE\_RESPONSE & S $\rightarrow$ C & Xác nhận chấm \\
\hline
GET\_REVIEW\_STATISTICS\_REQUEST & C $\rightarrow$ S & Thống kê chấm bài \\
\hline
GET\_REVIEW\_STATISTICS\_RESPONSE & S $\rightarrow$ C & Dữ liệu thống kê \\
\hline
\multicolumn{3}{|c|}{\textbf{Notifications}} \\
\hline
NEW\_SUBMISSION\_NOTIFICATION & S $\rightarrow$ C & Thông báo bài mới \\
\hline
EXERCISE\_FEEDBACK\_NOTIFICATION & S $\rightarrow$ C & Thông báo có phản hồi \\
\hline
\caption{Danh sách bản tin Exercise}
\end{longtable}

\subsection{Chi Tiết Bản Tin}

\subsubsection{GET\_EXERCISE\_LIST\_REQUEST}

\begin{lstlisting}[caption={GET\_EXERCISE\_LIST\_REQUEST}]
{
    "messageType": "GET_EXERCISE_LIST_REQUEST",
    "sessionToken": "token_xyz789",
    "payload": {
        "exerciseType": "writing",
        "level": "beginner"
    }
}
\end{lstlisting}

\subsubsection{SUBMIT\_EXERCISE\_REQUEST}

\begin{lstlisting}[caption={SUBMIT\_EXERCISE\_REQUEST}]
{
    "messageType": "SUBMIT_EXERCISE_REQUEST",
    "sessionToken": "token_xyz789",
    "payload": {
        "exerciseId": "ex_001",
        "exerciseType": "writing",
        "content": "I wake up at 6 AM every day...",
        "audioUrl": ""
    }
}
\end{lstlisting}

\subsubsection{REVIEW\_EXERCISE\_REQUEST}

\begin{lstlisting}[caption={REVIEW\_EXERCISE\_REQUEST (Teacher)}]
{
    "messageType": "REVIEW_EXERCISE_REQUEST",
    "sessionToken": "teacher_token",
    "payload": {
        "submissionId": "sub_001",
        "score": 85,
        "feedback": "Good work! Try to use more varied vocabulary."
    }
}
\end{lstlisting}

\subsubsection{EXERCISE\_FEEDBACK\_NOTIFICATION}

\begin{lstlisting}[caption={EXERCISE\_FEEDBACK\_NOTIFICATION}]
{
    "messageType": "EXERCISE_FEEDBACK_NOTIFICATION",
    "timestamp": 1704067200000,
    "payload": {
        "exerciseId": "ex_001",
        "exerciseTitle": "My Daily Routine",
        "score": 85,
        "feedback": "Good work!",
        "reviewedBy": "Teacher Sarah"
    }
}
\end{lstlisting}

\newpage
%====================================================================
\section{Chat với User Khác}
%====================================================================

\subsection{Mô Tả Chức Năng}

Chức năng chat cho phép:
\begin{itemize}
    \item Xem danh sách người dùng (online/offline)
    \item Gửi tin nhắn cho người dùng khác
    \item Nhận tin nhắn real-time
    \item Xem lịch sử chat
    \item Đánh dấu tin nhắn đã đọc
\end{itemize}

\subsection{Quy Trình Chat}

\begin{figure}[H]
\centering
\begin{tikzpicture}[node distance=1.3cm, scale=0.8, transform shape]
    % Nodes
    \node (start) [startstop] {Chọn "Chat"};
    \node (send1) [process, below of=start] {GET\_CONTACT\_LIST\_REQUEST};
    \node (wait1) [process, below of=send1] {GET\_CONTACT\_LIST\_RESPONSE};
    \node (display) [io, below of=wait1] {Hiển thị danh sách user};
    \node (select) [process, below of=display] {Chọn user để chat};
    \node (send2) [process, below of=select] {GET\_CHAT\_HISTORY\_REQUEST};
    \node (wait2) [process, below of=send2] {GET\_CHAT\_HISTORY\_RESPONSE};
    \node (showChat) [io, below of=wait2] {Hiển thị lịch sử};
    \node (loop) [decision, below of=showChat, yshift=-0.5cm] {Tiếp tục?};
    \node (input) [io, below of=loop, yshift=-0.5cm, xshift=-3cm] {Nhập tin nhắn};
    \node (send3) [process, below of=input] {SEND\_MESSAGE\_REQUEST};
    \node (end) [startstop, below of=loop, yshift=-3cm] {Kết thúc};

    % Arrows
    \draw [arrow] (start) -- (send1);
    \draw [arrow] (send1) -- (wait1);
    \draw [arrow] (wait1) -- (display);
    \draw [arrow] (display) -- (select);
    \draw [arrow] (select) -- (send2);
    \draw [arrow] (send2) -- (wait2);
    \draw [arrow] (wait2) -- (showChat);
    \draw [arrow] (showChat) -- (loop);
    \draw [arrow] (loop) -| node[above left, pos=0.3] {Có} (input);
    \draw [arrow] (input) -- (send3);
    \draw [arrow] (send3) |- (showChat);
    \draw [arrow] (loop) -- node[right] {Không} (end);
\end{tikzpicture}
\caption{Flowchart quy trình chat}
\end{figure}

\subsection{Sequence Diagram - Gửi/Nhận Tin Nhắn}

\begin{figure}[H]
\centering
\begin{tikzpicture}[scale=0.85, transform shape]
    % Lifelines
    \node[client] (userA) at (0,0) {User A};
    \node[server] (server) at (6,0) {Server};
    \node[client] (userB) at (12,0) {User B};

    \draw[dashed] (0,-0.5) -- (0,-14);
    \draw[dashed] (6,-0.5) -- (6,-14);
    \draw[dashed] (12,-0.5) -- (12,-14);

    % Get contact list
    \draw[->, thick, clientcolor] (0,-1.5) -- node[above, font=\scriptsize] {GET\_CONTACT\_LIST\_REQUEST} (6,-1.5);
    \draw[<-, thick, servercolor] (0,-2.5) -- node[above, font=\scriptsize] {GET\_CONTACT\_LIST\_RESPONSE} (6,-2.5);

    % Get chat history
    \draw[->, thick, clientcolor] (0,-3.5) -- node[above, font=\scriptsize] {GET\_CHAT\_HISTORY\_REQUEST} (6,-3.5);
    \draw[<-, thick, servercolor] (0,-4.5) -- node[above, font=\scriptsize] {GET\_CHAT\_HISTORY\_RESPONSE} (6,-4.5);

    % Send message
    \draw[->, thick, clientcolor] (0,-6) -- node[above, font=\scriptsize] {SEND\_MESSAGE\_REQUEST} (6,-6);

    % Server processing
    \fill[servercolor!30] (5.85,-6) rectangle (6.15,-7.5);
    \node[font=\scriptsize, align=left, anchor=west] at (6.3,-6.75) {Lưu tin nhắn\\Tìm recipient};

    \draw[<-, thick, servercolor] (0,-7.5) -- node[above, font=\scriptsize] {SEND\_MESSAGE\_RESPONSE} (6,-7.5);

    % Notification to User B
    \draw[->, thick, notifycolor] (6,-8.5) -- node[above, font=\scriptsize] {RECEIVE\_MESSAGE} (12,-8.5);

    % User B replies
    \draw[->, thick, clientcolor] (12,-10) -- node[above, font=\scriptsize] {SEND\_MESSAGE\_REQUEST} (6,-10);
    \draw[<-, thick, servercolor] (12,-11) -- node[above, font=\scriptsize] {SEND\_MESSAGE\_RESPONSE} (6,-11);

    % Notification to User A
    \draw[->, thick, notifycolor] (6,-12) -- node[above, font=\scriptsize] {RECEIVE\_MESSAGE} (0,-12);

    % Mark read
    \draw[->, thick, clientcolor] (0,-13) -- node[above, font=\scriptsize] {MARK\_MESSAGES\_READ\_REQUEST} (6,-13);
\end{tikzpicture}
\caption{Sequence diagram chat giữa 2 user}
\end{figure}

\subsection{Danh Sách Bản Tin}

\begin{longtable}{|p{5.5cm}|p{2.5cm}|p{5.5cm}|}
\hline
\textbf{Message Type} & \textbf{Hướng} & \textbf{Mô tả} \\
\hline
\endhead
GET\_CONTACT\_LIST\_REQUEST & C $\rightarrow$ S & Lấy danh sách liên hệ \\
\hline
GET\_CONTACT\_LIST\_RESPONSE & S $\rightarrow$ C & Danh sách user + trạng thái \\
\hline
SEND\_MESSAGE\_REQUEST & C $\rightarrow$ S & Gửi tin nhắn \\
\hline
SEND\_MESSAGE\_RESPONSE & S $\rightarrow$ C & Xác nhận gửi \\
\hline
GET\_CHAT\_HISTORY\_REQUEST & C $\rightarrow$ S & Lấy lịch sử chat \\
\hline
GET\_CHAT\_HISTORY\_RESPONSE & S $\rightarrow$ C & Danh sách tin nhắn \\
\hline
MARK\_MESSAGES\_READ\_REQUEST & C $\rightarrow$ S & Đánh dấu đã đọc \\
\hline
MARK\_MESSAGES\_READ\_RESPONSE & S $\rightarrow$ C & Xác nhận \\
\hline
\multicolumn{3}{|c|}{\textbf{Notifications}} \\
\hline
RECEIVE\_MESSAGE & S $\rightarrow$ C & Tin nhắn mới (real-time) \\
\hline
UNREAD\_MESSAGES\_NOTIFICATION & S $\rightarrow$ C & Thông báo tin chưa đọc \\
\hline
\caption{Danh sách bản tin Chat}
\end{longtable}

\subsection{Chi Tiết Bản Tin}

\subsubsection{GET\_CONTACT\_LIST\_REQUEST}

\begin{lstlisting}[caption={GET\_CONTACT\_LIST\_REQUEST}]
{
    "messageType": "GET_CONTACT_LIST_REQUEST",
    "sessionToken": "token_xyz789",
    "payload": {}
}
\end{lstlisting}

\subsubsection{GET\_CONTACT\_LIST\_RESPONSE}

\begin{lstlisting}[caption={GET\_CONTACT\_LIST\_RESPONSE}]
{
    "messageType": "GET_CONTACT_LIST_RESPONSE",
    "payload": {
        "status": "success",
        "data": {
            "users": [
                {
                    "userId": "user_002",
                    "fullname": "Teacher Sarah",
                    "role": "teacher",
                    "online": true
                },
                {
                    "userId": "user_003",
                    "fullname": "Student Bob",
                    "role": "student",
                    "online": false
                }
            ]
        }
    }
}
\end{lstlisting}

\subsubsection{SEND\_MESSAGE\_REQUEST}

\begin{lstlisting}[caption={SEND\_MESSAGE\_REQUEST}]
{
    "messageType": "SEND_MESSAGE_REQUEST",
    "sessionToken": "token_xyz789",
    "payload": {
        "recipientId": "user_002",
        "messageContent": "Hello teacher, I have a question."
    }
}
\end{lstlisting}

\subsubsection{SEND\_MESSAGE\_RESPONSE}

\begin{lstlisting}[caption={SEND\_MESSAGE\_RESPONSE}]
{
    "messageType": "SEND_MESSAGE_RESPONSE",
    "payload": {
        "status": "success",
        "data": {
            "messageId": "chat_msg_001",
            "sentAt": 1704067200000
        }
    }
}
\end{lstlisting}

\subsubsection{RECEIVE\_MESSAGE (Notification)}

\begin{lstlisting}[caption={RECEIVE\_MESSAGE}]
{
    "messageType": "RECEIVE_MESSAGE",
    "timestamp": 1704067200000,
    "payload": {
        "senderId": "user_001",
        "senderName": "Nguyen Van A",
        "messageContent": "Hello teacher, I have a question.",
        "sentAt": 1704067200000
    }
}
\end{lstlisting}

\subsubsection{GET\_CHAT\_HISTORY\_REQUEST}

\begin{lstlisting}[caption={GET\_CHAT\_HISTORY\_REQUEST}]
{
    "messageType": "GET_CHAT_HISTORY_REQUEST",
    "sessionToken": "token_xyz789",
    "payload": {
        "senderId": "user_002"
    }
}
\end{lstlisting}

\subsubsection{GET\_CHAT\_HISTORY\_RESPONSE}

\begin{lstlisting}[caption={GET\_CHAT\_HISTORY\_RESPONSE}]
{
    "messageType": "GET_CHAT_HISTORY_RESPONSE",
    "payload": {
        "status": "success",
        "data": {
            "messages": [
                {
                    "messageId": "chat_001",
                    "senderId": "user_001",
                    "senderName": "Nguyen Van A",
                    "content": "Hello teacher!",
                    "timestamp": 1704060000000
                },
                {
                    "messageId": "chat_002",
                    "senderId": "user_002",
                    "senderName": "Teacher Sarah",
                    "content": "Hi! How can I help?",
                    "timestamp": 1704060060000
                }
            ]
        }
    }
}
\end{lstlisting}

\newpage
%====================================================================
\section{Chơi Game}
%====================================================================

\subsection{Mô Tả Chức Năng}

Chức năng chơi game bao gồm các loại game:
\begin{itemize}
    \item \textbf{Word Match}: Ghép từ tiếng Anh - tiếng Việt
    \item \textbf{Picture Match}: Ghép hình ảnh với từ vựng
    \item \textbf{Sentence Match}: Ghép câu với nghĩa
\end{itemize}

\subsection{Quy Trình Chơi Game}

\begin{figure}[H]
\centering
\begin{tikzpicture}[node distance=1.3cm, scale=0.8, transform shape]
    % Nodes
    \node (start) [startstop] {Chọn "Chơi game"};
    \node (send1) [process, below of=start] {GET\_GAME\_LIST\_REQUEST};
    \node (wait1) [process, below of=send1] {GET\_GAME\_LIST\_RESPONSE};
    \node (display) [io, below of=wait1] {Hiển thị danh sách game};
    \node (select) [process, below of=display] {Chọn game};
    \node (send2) [process, below of=select] {START\_GAME\_REQUEST};
    \node (wait2) [process, below of=send2] {START\_GAME\_RESPONSE};
    \node (play) [io, below of=wait2] {Chơi game};
    \node (timer) [decision, below of=play, yshift=-0.5cm] {Hết giờ?};
    \node (submit) [process, below of=timer, yshift=-0.5cm] {SUBMIT\_GAME\_RESULT\_REQUEST};
    \node (wait3) [process, below of=submit] {SUBMIT\_GAME\_RESULT\_RESPONSE};
    \node (result) [io, below of=wait3] {Hiển thị kết quả};
    \node (again) [decision, below of=result, yshift=-0.5cm] {Chơi lại?};
    \node (end) [startstop, right of=again, xshift=3cm] {Kết thúc};

    % Arrows
    \draw [arrow] (start) -- (send1);
    \draw [arrow] (send1) -- (wait1);
    \draw [arrow] (wait1) -- (display);
    \draw [arrow] (display) -- (select);
    \draw [arrow] (select) -- (send2);
    \draw [arrow] (send2) -- (wait2);
    \draw [arrow] (wait2) -- (play);
    \draw [arrow] (play) -- (timer);
    \draw [arrow] (timer) -- node[left] {Có/Xong} (submit);
    \draw [arrow] (timer.east) -- ++(1,0) |- node[right, pos=0.25] {Không} (play.east);
    \draw [arrow] (submit) -- (wait3);
    \draw [arrow] (wait3) -- (result);
    \draw [arrow] (result) -- (again);
    \draw [arrow] (again) -- node[above] {Không} (end);
    \draw [arrow] (again.west) -- ++(-1,0) |- node[left, pos=0.25] {Có} (select.west);
\end{tikzpicture}
\caption{Flowchart quy trình chơi game}
\end{figure}

\subsection{Sequence Diagram}

\begin{figure}[H]
\centering
\begin{tikzpicture}[scale=0.85, transform shape]
    % Lifelines
    \node[client] (client) at (0,0) {Client};
    \node[server] (server) at (8,0) {Server};

    \draw[dashed] (0,-0.5) -- (0,-14);
    \draw[dashed] (8,-0.5) -- (8,-14);

    % Get game list
    \draw[->, thick, clientcolor] (0,-1.5) -- node[above, font=\scriptsize] {GET\_GAME\_LIST\_REQUEST} (8,-1.5);
    \draw[<-, thick, servercolor] (0,-2.5) -- node[above, font=\scriptsize] {GET\_GAME\_LIST\_RESPONSE} (8,-2.5);

    % Start game
    \draw[->, thick, clientcolor] (0,-4) -- node[above, font=\scriptsize] {START\_GAME\_REQUEST} (8,-4);

    % Server prepares game
    \fill[servercolor!30] (7.85,-4) rectangle (8.15,-5.5);
    \node[font=\scriptsize, align=left, anchor=west] at (8.3,-4.75) {Tạo game session\\Shuffle pairs};

    \draw[<-, thick, servercolor] (0,-6) -- node[above, font=\scriptsize] {START\_GAME\_RESPONSE} (8,-6);

    % User plays
    \draw[dashed] (-1,-7) rectangle (1,-9.5);
    \node[font=\scriptsize] at (0,-7.3) {loop [Chơi game]};
    \node[font=\scriptsize, align=center] at (0,-8.5) {Load hình ảnh\\User ghép cặp\\Đếm giờ};

    % Submit result
    \draw[->, thick, clientcolor] (0,-10.5) -- node[above, font=\scriptsize] {SUBMIT\_GAME\_RESULT\_REQUEST} (8,-10.5);

    % Server calculates score
    \fill[servercolor!30] (7.85,-10.5) rectangle (8.15,-12);
    \node[font=\scriptsize, align=left, anchor=west] at (8.3,-11.25) {Tính điểm\\Lưu kết quả};

    \draw[<-, thick, servercolor] (0,-12.5) -- node[above, font=\scriptsize] {SUBMIT\_GAME\_RESULT\_RESPONSE} (8,-12.5);
\end{tikzpicture}
\caption{Sequence diagram chơi game}
\end{figure}

\subsection{Danh Sách Bản Tin}

\begin{longtable}{|p{5.5cm}|p{2.5cm}|p{5.5cm}|}
\hline
\textbf{Message Type} & \textbf{Hướng} & \textbf{Mô tả} \\
\hline
\endhead
GET\_GAME\_LIST\_REQUEST & C $\rightarrow$ S & Lấy danh sách game \\
\hline
GET\_GAME\_LIST\_RESPONSE & S $\rightarrow$ C & Danh sách game có sẵn \\
\hline
START\_GAME\_REQUEST & C $\rightarrow$ S & Bắt đầu game \\
\hline
START\_GAME\_RESPONSE & S $\rightarrow$ C & Dữ liệu game (pairs) \\
\hline
SUBMIT\_GAME\_RESULT\_REQUEST & C $\rightarrow$ S & Nộp kết quả game \\
\hline
SUBMIT\_GAME\_RESULT\_RESPONSE & S $\rightarrow$ C & Điểm số và feedback \\
\hline
\multicolumn{3}{|c|}{\textbf{Admin (Quản lý game)}} \\
\hline
ADD\_GAME\_REQUEST & C $\rightarrow$ S & Thêm game mới \\
\hline
ADD\_GAME\_RESPONSE & S $\rightarrow$ C & Xác nhận thêm \\
\hline
UPDATE\_GAME\_REQUEST & C $\rightarrow$ S & Cập nhật game \\
\hline
UPDATE\_GAME\_RESPONSE & S $\rightarrow$ C & Xác nhận cập nhật \\
\hline
DELETE\_GAME\_REQUEST & C $\rightarrow$ S & Xóa game \\
\hline
DELETE\_GAME\_RESPONSE & S $\rightarrow$ C & Xác nhận xóa \\
\hline
GET\_ADMIN\_GAMES\_REQUEST & C $\rightarrow$ S & Lấy tất cả game (Admin) \\
\hline
GET\_ADMIN\_GAMES\_RESPONSE & S $\rightarrow$ C & Danh sách đầy đủ \\
\hline
\caption{Danh sách bản tin Game}
\end{longtable}

\subsection{Chi Tiết Bản Tin}

\subsubsection{GET\_GAME\_LIST\_REQUEST}

\begin{lstlisting}[caption={GET\_GAME\_LIST\_REQUEST}]
{
    "messageType": "GET_GAME_LIST_REQUEST",
    "sessionToken": "token_xyz789",
    "payload": {
        "gameType": "all",
        "level": "beginner"
    }
}
\end{lstlisting}

\subsubsection{GET\_GAME\_LIST\_RESPONSE}

\begin{lstlisting}[caption={GET\_GAME\_LIST\_RESPONSE}]
{
    "messageType": "GET_GAME_LIST_RESPONSE",
    "payload": {
        "status": "success",
        "data": {
            "games": [
                {
                    "gameId": "game_001",
                    "title": "Word Match",
                    "gameType": "word_match",
                    "level": "beginner",
                    "description": "Match English-Vietnamese words"
                },
                {
                    "gameId": "game_004",
                    "title": "Fruit Pictures",
                    "gameType": "picture_match",
                    "level": "beginner",
                    "description": "Match words with fruit images"
                }
            ]
        }
    }
}
\end{lstlisting}

\subsubsection{START\_GAME\_REQUEST}

\begin{lstlisting}[caption={START\_GAME\_REQUEST}]
{
    "messageType": "START_GAME_REQUEST",
    "sessionToken": "token_xyz789",
    "payload": {
        "gameId": "game_004"
    }
}
\end{lstlisting}

\subsubsection{START\_GAME\_RESPONSE (Word Match)}

\begin{lstlisting}[caption={START\_GAME\_RESPONSE - Word Match}]
{
    "messageType": "START_GAME_RESPONSE",
    "payload": {
        "status": "success",
        "data": {
            "gameSessionId": "gs_002",
            "gameId": "game_001",
            "gameType": "word_match",
            "title": "Word Match",
            "timeLimit": 180,
            "pairs": [
                {"left": "Hello", "right": "Xin chao"},
                {"left": "Goodbye", "right": "Tam biet"},
                {"left": "Thank you", "right": "Cam on"}
            ]
        }
    }
}
\end{lstlisting}

\subsubsection{START\_GAME\_RESPONSE (Picture Match)}

\begin{lstlisting}[caption={START\_GAME\_RESPONSE - Picture Match}]
{
    "messageType": "START_GAME_RESPONSE",
    "payload": {
        "status": "success",
        "data": {
            "gameSessionId": "gs_001",
            "gameId": "game_004",
            "gameType": "picture_match",
            "title": "Fruit Pictures",
            "timeLimit": 120,
            "maxScore": 100,
            "pairs": [
                {
                    "word": "Apple",
                    "imageUrl": "https://cdn-icons-png.flaticon.com/128/415/415682.png"
                },
                {
                    "word": "Banana",
                    "imageUrl": "https://cdn-icons-png.flaticon.com/128/3143/3143643.png"
                },
                {
                    "word": "Orange",
                    "imageUrl": "https://cdn-icons-png.flaticon.com/128/415/415733.png"
                }
            ]
        }
    }
}
\end{lstlisting}

\subsubsection{SUBMIT\_GAME\_RESULT\_REQUEST}

\begin{lstlisting}[caption={SUBMIT\_GAME\_RESULT\_REQUEST}]
{
    "messageType": "SUBMIT_GAME_RESULT_REQUEST",
    "sessionToken": "token_xyz789",
    "payload": {
        "gameSessionId": "gs_001",
        "gameId": "game_004",
        "matches": [
            {"word": "Apple", "imageUrl": "https://...415682.png"},
            {"word": "Banana", "imageUrl": "https://...3143643.png"},
            {"word": "Orange", "imageUrl": "https://...415733.png"}
        ]
    }
}
\end{lstlisting}

\subsubsection{SUBMIT\_GAME\_RESULT\_RESPONSE}

\begin{lstlisting}[caption={SUBMIT\_GAME\_RESULT\_RESPONSE}]
{
    "messageType": "SUBMIT_GAME_RESULT_RESPONSE",
    "payload": {
        "status": "success",
        "data": {
            "score": 100,
            "maxScore": 100,
            "correctMatches": 3,
            "totalPairs": 3,
            "timeUsed": 45,
            "feedback": "Excellent! Perfect score!"
        }
    }
}
\end{lstlisting}

\newpage
%====================================================================
\section{Voice Call (Gọi Thoại)}
%====================================================================

\subsection{Mô Tả Chức Năng}

Chức năng gọi thoại cho phép:
\begin{itemize}
    \item Khởi tạo cuộc gọi đến user khác
    \item Nhận thông báo cuộc gọi đến
    \item Chấp nhận hoặc từ chối cuộc gọi
    \item Kết thúc cuộc gọi
\end{itemize}

\subsection{Sequence Diagram - Voice Call}

\begin{figure}[H]
\centering
\begin{tikzpicture}[scale=0.75, transform shape]
    % Lifelines
    \node[client] (caller) at (0,0) {Caller};
    \node[server] (server) at (6,0) {Server};
    \node[client] (receiver) at (12,0) {Receiver};

    \draw[dashed] (0,-0.5) -- (0,-16);
    \draw[dashed] (6,-0.5) -- (6,-16);
    \draw[dashed] (12,-0.5) -- (12,-16);

    % Initiate call
    \draw[->, thick, clientcolor] (0,-1.5) -- node[above, font=\scriptsize] {VOICE\_CALL\_INITIATE\_REQUEST} (6,-1.5);
    \draw[<-, thick, servercolor] (0,-2.5) -- node[above, font=\scriptsize] {VOICE\_CALL\_INITIATE\_RESPONSE} (6,-2.5);

    % Notification to receiver
    \draw[->, thick, notifycolor] (6,-3.5) -- node[above, font=\scriptsize] {VOICE\_CALL\_INCOMING} (12,-3.5);

    % Alternative: Accept or Reject
    \draw[dashed] (-1,-4.5) rectangle (13,-9);
    \node[font=\scriptsize] at (1,-4.8) {alt [Accept]};

    \draw[->, thick, clientcolor] (12,-5.5) -- node[above, font=\scriptsize] {VOICE\_CALL\_ACCEPT\_REQUEST} (6,-5.5);
    \draw[<-, thick, servercolor] (12,-6.5) -- node[above, font=\scriptsize] {VOICE\_CALL\_ACCEPT\_RESPONSE} (6,-6.5);
    \draw[->, thick, notifycolor] (6,-7) -- node[above, font=\scriptsize] {VOICE\_CALL\_ACCEPTED} (0,-7);

    \draw[dashed] (-0.5,-7.5) -- (12.5,-7.5);
    \node[font=\scriptsize] at (1,-7.8) {[Reject]};
    \draw[->, thick, red] (12,-8.5) -- node[above, font=\scriptsize] {VOICE\_CALL\_REJECT\_REQUEST} (6,-8.5);

    % Call in progress
    \draw[dashed] (-1,-10) rectangle (13,-12);
    \node[font=\scriptsize] at (0,-10.3) {loop [In Call]};
    \node[font=\scriptsize, align=center] at (6,-11) {Cuộc gọi đang diễn ra\\(Audio streaming)};

    % End call
    \draw[->, thick, clientcolor] (0,-13) -- node[above, font=\scriptsize] {VOICE\_CALL\_END\_REQUEST} (6,-13);
    \draw[<-, thick, servercolor] (0,-14) -- node[above, font=\scriptsize] {VOICE\_CALL\_END\_RESPONSE} (6,-14);
    \draw[->, thick, notifycolor] (6,-15) -- node[above, font=\scriptsize] {VOICE\_CALL\_ENDED} (12,-15);
\end{tikzpicture}
\caption{Sequence diagram voice call}
\end{figure}

\subsection{Danh Sách Bản Tin}

\begin{longtable}{|p{5.5cm}|p{2.5cm}|p{5.5cm}|}
\hline
\textbf{Message Type} & \textbf{Hướng} & \textbf{Mô tả} \\
\hline
\endhead
VOICE\_CALL\_INITIATE\_REQUEST & C $\rightarrow$ S & Khởi tạo cuộc gọi \\
\hline
VOICE\_CALL\_INITIATE\_RESPONSE & S $\rightarrow$ C & Kết quả khởi tạo \\
\hline
VOICE\_CALL\_ACCEPT\_REQUEST & C $\rightarrow$ S & Chấp nhận cuộc gọi \\
\hline
VOICE\_CALL\_ACCEPT\_RESPONSE & S $\rightarrow$ C & Xác nhận chấp nhận \\
\hline
VOICE\_CALL\_REJECT\_REQUEST & C $\rightarrow$ S & Từ chối cuộc gọi \\
\hline
VOICE\_CALL\_REJECT\_RESPONSE & S $\rightarrow$ C & Xác nhận từ chối \\
\hline
VOICE\_CALL\_END\_REQUEST & C $\rightarrow$ S & Kết thúc cuộc gọi \\
\hline
VOICE\_CALL\_END\_RESPONSE & S $\rightarrow$ C & Xác nhận kết thúc \\
\hline
VOICE\_CALL\_GET\_STATUS\_REQUEST & C $\rightarrow$ S & Lấy trạng thái cuộc gọi \\
\hline
VOICE\_CALL\_GET\_STATUS\_RESPONSE & S $\rightarrow$ C & Trạng thái hiện tại \\
\hline
\multicolumn{3}{|c|}{\textbf{Notifications}} \\
\hline
VOICE\_CALL\_INCOMING & S $\rightarrow$ C & Báo có cuộc gọi đến \\
\hline
VOICE\_CALL\_ACCEPTED & S $\rightarrow$ C & Cuộc gọi được chấp nhận \\
\hline
VOICE\_CALL\_REJECTED & S $\rightarrow$ C & Cuộc gọi bị từ chối \\
\hline
VOICE\_CALL\_ENDED & S $\rightarrow$ C & Cuộc gọi kết thúc \\
\hline
\caption{Danh sách bản tin Voice Call}
\end{longtable}

\newpage
%====================================================================
\section{Tổng Hợp Bản Tin}
%====================================================================

\subsection{Bảng Tổng Hợp Tất Cả Bản Tin}

\begin{longtable}{|p{1cm}|p{5cm}|p{2cm}|p{5.5cm}|}
\hline
\textbf{STT} & \textbf{Message Type} & \textbf{Hướng} & \textbf{Chức năng} \\
\hline
\endhead
\multicolumn{4}{|c|}{\textbf{AUTHENTICATION (6 bản tin)}} \\
\hline
1 & REGISTER\_REQUEST & C $\rightarrow$ S & Đăng ký \\
2 & REGISTER\_RESPONSE & S $\rightarrow$ C & Đăng ký \\
3 & LOGIN\_REQUEST & C $\rightarrow$ S & Đăng nhập \\
4 & LOGIN\_RESPONSE & S $\rightarrow$ C & Đăng nhập \\
5 & SET\_LEVEL\_REQUEST & C $\rightarrow$ S & Đăng nhập \\
6 & SET\_LEVEL\_RESPONSE & S $\rightarrow$ C & Đăng nhập \\
\hline
\multicolumn{4}{|c|}{\textbf{LESSON (4 bản tin)}} \\
\hline
7 & GET\_LESSONS\_REQUEST & C $\rightarrow$ S & Học bài \\
8 & GET\_LESSONS\_RESPONSE & S $\rightarrow$ C & Học bài \\
9 & GET\_LESSON\_DETAIL\_REQUEST & C $\rightarrow$ S & Học bài \\
10 & GET\_LESSON\_DETAIL\_RESPONSE & S $\rightarrow$ C & Học bài \\
\hline
\multicolumn{4}{|c|}{\textbf{TEST (4 bản tin)}} \\
\hline
11 & GET\_TEST\_REQUEST & C $\rightarrow$ S & Làm bài kiểm tra \\
12 & GET\_TEST\_RESPONSE & S $\rightarrow$ C & Làm bài kiểm tra \\
13 & SUBMIT\_TEST\_REQUEST & C $\rightarrow$ S & Làm bài kiểm tra \\
14 & SUBMIT\_TEST\_RESPONSE & S $\rightarrow$ C & Làm bài kiểm tra \\
\hline
\multicolumn{4}{|c|}{\textbf{EXERCISE (24 bản tin)}} \\
\hline
15 & GET\_EXERCISE\_LIST\_REQUEST & C $\rightarrow$ S & Làm bài tập \\
16 & GET\_EXERCISE\_LIST\_RESPONSE & S $\rightarrow$ C & Làm bài tập \\
17 & GET\_EXERCISE\_REQUEST & C $\rightarrow$ S & Làm bài tập \\
18 & GET\_EXERCISE\_RESPONSE & S $\rightarrow$ C & Làm bài tập \\
19 & SAVE\_DRAFT\_REQUEST & C $\rightarrow$ S & Làm bài tập \\
20 & SAVE\_DRAFT\_RESPONSE & S $\rightarrow$ C & Làm bài tập \\
21 & SUBMIT\_EXERCISE\_REQUEST & C $\rightarrow$ S & Làm bài tập \\
22 & SUBMIT\_EXERCISE\_RESPONSE & S $\rightarrow$ C & Làm bài tập \\
23 & GET\_USER\_SUBMISSIONS\_REQUEST & C $\rightarrow$ S & Làm bài tập \\
24 & GET\_USER\_SUBMISSIONS\_RESPONSE & S $\rightarrow$ C & Làm bài tập \\
25 & GET\_FEEDBACK\_REQUEST & C $\rightarrow$ S & Làm bài tập \\
26 & GET\_FEEDBACK\_RESPONSE & S $\rightarrow$ C & Làm bài tập \\
27 & GET\_MY\_DRAFTS\_REQUEST & C $\rightarrow$ S & Làm bài tập \\
28 & GET\_MY\_DRAFTS\_RESPONSE & S $\rightarrow$ C & Làm bài tập \\
29 & GET\_PENDING\_REVIEWS\_REQUEST & C $\rightarrow$ S & Chấm bài (Teacher) \\
30 & GET\_PENDING\_REVIEWS\_RESPONSE & S $\rightarrow$ C & Chấm bài (Teacher) \\
31 & GET\_SUBMISSION\_DETAIL\_REQUEST & C $\rightarrow$ S & Chấm bài (Teacher) \\
32 & GET\_SUBMISSION\_DETAIL\_RESPONSE & S $\rightarrow$ C & Chấm bài (Teacher) \\
33 & REVIEW\_EXERCISE\_REQUEST & C $\rightarrow$ S & Chấm bài (Teacher) \\
34 & REVIEW\_EXERCISE\_RESPONSE & S $\rightarrow$ C & Chấm bài (Teacher) \\
35 & GET\_REVIEW\_STATISTICS\_REQUEST & C $\rightarrow$ S & Chấm bài (Teacher) \\
36 & GET\_REVIEW\_STATISTICS\_RESPONSE & S $\rightarrow$ C & Chấm bài (Teacher) \\
37 & NEW\_SUBMISSION\_NOTIFICATION & S $\rightarrow$ C & Thông báo \\
38 & EXERCISE\_FEEDBACK\_NOTIFICATION & S $\rightarrow$ C & Thông báo \\
\hline
\multicolumn{4}{|c|}{\textbf{CHAT (12 bản tin)}} \\
\hline
39 & GET\_CONTACT\_LIST\_REQUEST & C $\rightarrow$ S & Chat \\
40 & GET\_CONTACT\_LIST\_RESPONSE & S $\rightarrow$ C & Chat \\
41 & SEND\_MESSAGE\_REQUEST & C $\rightarrow$ S & Chat \\
42 & SEND\_MESSAGE\_RESPONSE & S $\rightarrow$ C & Chat \\
43 & GET\_CHAT\_HISTORY\_REQUEST & C $\rightarrow$ S & Chat \\
44 & GET\_CHAT\_HISTORY\_RESPONSE & S $\rightarrow$ C & Chat \\
45 & MARK\_MESSAGES\_READ\_REQUEST & C $\rightarrow$ S & Chat \\
46 & MARK\_MESSAGES\_READ\_RESPONSE & S $\rightarrow$ C & Chat \\
47 & RECEIVE\_MESSAGE & S $\rightarrow$ C & Chat (Notification) \\
48 & UNREAD\_MESSAGES\_NOTIFICATION & S $\rightarrow$ C & Chat (Notification) \\
\hline
\multicolumn{4}{|c|}{\textbf{GAME (14 bản tin)}} \\
\hline
49 & GET\_GAME\_LIST\_REQUEST & C $\rightarrow$ S & Chơi game \\
50 & GET\_GAME\_LIST\_RESPONSE & S $\rightarrow$ C & Chơi game \\
51 & START\_GAME\_REQUEST & C $\rightarrow$ S & Chơi game \\
52 & START\_GAME\_RESPONSE & S $\rightarrow$ C & Chơi game \\
53 & SUBMIT\_GAME\_RESULT\_REQUEST & C $\rightarrow$ S & Chơi game \\
54 & SUBMIT\_GAME\_RESULT\_RESPONSE & S $\rightarrow$ C & Chơi game \\
55 & ADD\_GAME\_REQUEST & C $\rightarrow$ S & Quản lý game (Admin) \\
56 & ADD\_GAME\_RESPONSE & S $\rightarrow$ C & Quản lý game (Admin) \\
57 & UPDATE\_GAME\_REQUEST & C $\rightarrow$ S & Quản lý game (Admin) \\
58 & UPDATE\_GAME\_RESPONSE & S $\rightarrow$ C & Quản lý game (Admin) \\
59 & DELETE\_GAME\_REQUEST & C $\rightarrow$ S & Quản lý game (Admin) \\
60 & DELETE\_GAME\_RESPONSE & S $\rightarrow$ C & Quản lý game (Admin) \\
61 & GET\_ADMIN\_GAMES\_REQUEST & C $\rightarrow$ S & Quản lý game (Admin) \\
62 & GET\_ADMIN\_GAMES\_RESPONSE & S $\rightarrow$ C & Quản lý game (Admin) \\
\hline
\multicolumn{4}{|c|}{\textbf{VOICE CALL (14 bản tin)}} \\
\hline
63 & VOICE\_CALL\_INITIATE\_REQUEST & C $\rightarrow$ S & Gọi thoại \\
64 & VOICE\_CALL\_INITIATE\_RESPONSE & S $\rightarrow$ C & Gọi thoại \\
65 & VOICE\_CALL\_ACCEPT\_REQUEST & C $\rightarrow$ S & Gọi thoại \\
66 & VOICE\_CALL\_ACCEPT\_RESPONSE & S $\rightarrow$ C & Gọi thoại \\
67 & VOICE\_CALL\_REJECT\_REQUEST & C $\rightarrow$ S & Gọi thoại \\
68 & VOICE\_CALL\_REJECT\_RESPONSE & S $\rightarrow$ C & Gọi thoại \\
69 & VOICE\_CALL\_END\_REQUEST & C $\rightarrow$ S & Gọi thoại \\
70 & VOICE\_CALL\_END\_RESPONSE & S $\rightarrow$ C & Gọi thoại \\
71 & VOICE\_CALL\_GET\_STATUS\_REQUEST & C $\rightarrow$ S & Gọi thoại \\
72 & VOICE\_CALL\_GET\_STATUS\_RESPONSE & S $\rightarrow$ C & Gọi thoại \\
73 & VOICE\_CALL\_INCOMING & S $\rightarrow$ C & Gọi thoại (Notification) \\
74 & VOICE\_CALL\_ACCEPTED & S $\rightarrow$ C & Gọi thoại (Notification) \\
75 & VOICE\_CALL\_REJECTED & S $\rightarrow$ C & Gọi thoại (Notification) \\
76 & VOICE\_CALL\_ENDED & S $\rightarrow$ C & Gọi thoại (Notification) \\
\hline
\multicolumn{4}{|c|}{\textbf{ERROR (1 bản tin)}} \\
\hline
77 & ERROR\_RESPONSE & S $\rightarrow$ C & Xử lý lỗi \\
\hline
\caption{Bảng tổng hợp tất cả 77 bản tin}
\end{longtable}

\subsection{Thống Kê Theo Chức Năng}

\begin{table}[H]
\centering
\begin{tabular}{|l|c|c|c|c|}
\hline
\textbf{Chức năng} & \textbf{Request} & \textbf{Response} & \textbf{Notification} & \textbf{Tổng} \\
\hline
Authentication & 3 & 3 & 0 & 6 \\
Lesson & 2 & 2 & 0 & 4 \\
Test & 2 & 2 & 0 & 4 \\
Exercise & 11 & 11 & 2 & 24 \\
Chat & 4 & 4 & 2 & 10 \\
Game & 7 & 7 & 0 & 14 \\
Voice Call & 5 & 5 & 4 & 14 \\
Error & 0 & 1 & 0 & 1 \\
\hline
\textbf{Tổng cộng} & \textbf{34} & \textbf{35} & \textbf{8} & \textbf{77} \\
\hline
\end{tabular}
\caption{Thống kê số lượng bản tin theo chức năng}
\end{table}

\begin{figure}[H]
\centering
\begin{tikzpicture}
    % Bar chart
    \begin{scope}[xshift=0cm]
        \draw[->] (0,0) -- (9,0) node[right] {Chức năng};
        \draw[->] (0,0) -- (0,5) node[above] {Số bản tin};

        % Bars
        \fill[clientcolor!70] (0.5,0) rectangle (1.2,1.2);
        \node[font=\tiny] at (0.85,-0.3) {Auth};
        \node[font=\tiny] at (0.85,1.4) {6};

        \fill[servercolor!70] (1.5,0) rectangle (2.2,0.8);
        \node[font=\tiny] at (1.85,-0.3) {Lesson};
        \node[font=\tiny] at (1.85,1) {4};

        \fill[msgcolor!70] (2.5,0) rectangle (3.2,0.8);
        \node[font=\tiny] at (2.85,-0.3) {Test};
        \node[font=\tiny] at (2.85,1) {4};

        \fill[notifycolor!70] (3.5,0) rectangle (4.2,4.8);
        \node[font=\tiny] at (3.85,-0.3) {Exercise};
        \node[font=\tiny] at (3.85,5) {24};

        \fill[red!50] (4.5,0) rectangle (5.2,2);
        \node[font=\tiny] at (4.85,-0.3) {Chat};
        \node[font=\tiny] at (4.85,2.2) {10};

        \fill[blue!50] (5.5,0) rectangle (6.2,2.8);
        \node[font=\tiny] at (5.85,-0.3) {Game};
        \node[font=\tiny] at (5.85,3) {14};

        \fill[orange!50] (6.5,0) rectangle (7.2,2.8);
        \node[font=\tiny] at (6.85,-0.3) {Voice};
        \node[font=\tiny] at (6.85,3) {14};

        % Y-axis labels
        \foreach \y in {1,2,3,4,5} {
            \draw (-0.1,\y) -- (0.1,\y);
            \node[left, font=\tiny] at (-0.1,\y) {\pgfmathparse{int(\y*5)}\pgfmathresult};
        }
    \end{scope}
\end{tikzpicture}
\caption{Biểu đồ số lượng bản tin theo chức năng}
\end{figure}

\newpage
%====================================================================
\section*{Phụ Lục: Mã Lỗi}
%====================================================================
\addcontentsline{toc}{section}{Phụ Lục: Mã Lỗi}

\begin{table}[H]
\centering
\begin{tabular}{|l|l|p{7cm}|}
\hline
\textbf{Error Code} & \textbf{HTTP tương đương} & \textbf{Mô tả} \\
\hline
AUTH\_001 & 401 & Invalid credentials (sai email/password) \\
AUTH\_002 & 401 & Session expired (token hết hạn) \\
AUTH\_003 & 403 & Permission denied (không có quyền) \\
DATA\_001 & 404 & Resource not found (không tìm thấy) \\
DATA\_002 & 400 & Invalid data format (dữ liệu sai định dạng) \\
DATA\_003 & 409 & Duplicate entry (dữ liệu trùng lặp) \\
SYS\_001 & 500 & Internal server error (lỗi server) \\
SYS\_002 & 503 & Service unavailable (dịch vụ không khả dụng) \\
\hline
\end{tabular}
\caption{Bảng mã lỗi}
\end{table}

\begin{lstlisting}[caption={ERROR\_RESPONSE format}]
{
    "messageType": "ERROR_RESPONSE",
    "timestamp": 1704067200000,
    "payload": {
        "status": "error",
        "errorCode": "AUTH_001",
        "message": "Invalid email or password"
    }
}
\end{lstlisting}

\vspace{2cm}
\begin{center}
\rule{0.5\textwidth}{0.4pt}

\textit{Tài liệu quy trình và bản tin giao thức}

\textit{English Learning Application}

\textit{\today}
\end{center}

\end{document}
